%----------------------------------------------------------------------------
\appendix
%----------------------------------------------------------------------------
\chapter*{\fuggelek}\addcontentsline{toc}{chapter}{\fuggelek}
\setcounter{chapter}{\appendixnumber}
%\setcounter{equation}{0} % a fofejezet-szamlalo az angol ABC 6. betuje (F) lesz

%\numberwithin{tabular}{section}

\begin{lstlisting}[caption=The Mealy machine seen in Fig.\ref{fig:coffeemealy} in the form of the Xtext the grammar described in Listing \ref{li:xtext}.,label=li:coffeemealy]
MealyMachine{
initialState 
State a states { State a, State b, State c, State d, State e, State dd, State f
}inputAlphabet Alphabet { characters { water , pod , button , clean } }
outputAlphabet Alphabet { characters { done , coffee , none } }
transitions{ 
Transition { input clean output done sourceState a targetState a } , 
Transition { input pod output done sourceState a targetState b } , 
Transition { input water output done sourceState a targetState c } , 
Transition { input button output none sourceState a targetState f } , 
Transition { input pod output done sourceState b targetState b } , 
Transition { input water output done sourceState b targetState d } , 
Transition { input button output none sourceState b targetState f } , 
Transition { input clean output done sourceState b targetState a } , 
Transition { input clean output done sourceState c targetState a } , 
Transition { input pod output done sourceState c targetState dd } , 
Transition { input button output none sourceState c targetState f } , 
Transition { input water output done sourceState c targetState c } , 
Transition { input water output done sourceState d targetState d } , 
Transition { input pod output done sourceState d targetState d } , 
Transition { input clean output done sourceState d targetState a } , 
Transition { input button output coffee sourceState d targetState e } , 
Transition { input water output done sourceState dd targetState dd } , 
Transition { input pod output done sourceState dd targetState dd } , 
Transition { input button output coffee sourceState dd targetState e } , 
Transition { input clean output done sourceState dd targetState a } , 
Transition { input clean output done sourceState e targetState a } , 
Transition { input button output none sourceState e targetState f } , 
Transition { input pod output none sourceState e targetState f } , 
Transition { input water output none sourceState e targetState f } , 
Transition { input clean output none sourceState f targetState f } , 
Transition { input button output none sourceState f targetState f } , 
Transition { input pod output none sourceState f targetState f } , 
Transition { input water output none sourceState f targetState f } } }
\end{lstlisting}

\begin{figure}[H]
	\centering
	\begin{subfigure}{0.5\linewidth}
		\centering
		\includegraphics[width=1.0\linewidth]{figures/alternatingbit}
		\caption{Mealy machine representation of the alternating-bit protocol.}
	\end{subfigure}
	\begin{subfigure}{0.8\linewidth}
		\centering
		\begin{lstlisting}
			String sequence = 
				"null|send0|ack1|send0|ack0|send1"
				+ "|ack0null|send1|ack0ack0|send1|ack0ack1|send0"
				+ "|ack1null|send0|ack1ack0|send1|ack1ack1|send0"
				+ "|nullnull|send0|nullack0|send1|nullack1|send0"
				+ "|ack0nullnull|send1|ack0nullack0|send1|ack0nullack1|send0"
				+ "|ack0ack0null|send1|ack0ack0ack0|send1|ack0ack0ack1|send0"
				+ "|ack0ack1null|send0|ack0ack1ack0|send1|ack0ack1ack1|send0"
				+ "|ack1nullnull|send0|ack1nullack0|send1|ack1nullack1|send0"
				+ "|ack1ack0null|send1|ack1ack0ack0|send1|ack1ack0ack1|send0"
				+ "|ack1ack1null|send0|ack1ack1ack0|send1|ack1ack1ack1|send0"
				+ "|nullnullnull|send0|nullnullack0|send1|nullnullack1|send0"
				+ "|nullack0null|send1|nullack0ack0|send1|nullack0ack1|send0"
				+ "|nullack1null|send0|nullack1ack0|send1|nullack1ack1|send0";
		\end{lstlisting}
		\caption{}
	\end{subfigure}
	\caption{A Mealy machine \textbf{(a)}, with its input format \textbf{(b)} using the formalism described in Section \ref{item:stringsequencelearnable}s \emph{StringSequenceLearnable} item.}
	\label{fig:alternatingbit}
\end{figure}


\begin{lstlisting}[caption=Brute-force implementation of equivalence queries described in Section \ref{item:stringsequencelearnable}s \emph{StringSequenceAdapter} item using google guavas Sets.powerSet() function.,label=li:eqbruteforce]
@Override
public List<? extends I> equivalenceQuery(H hypothesis, Collection<? extends I> alphabet) {
	for(Set<I> s : com.google.common.collect.Sets.powerSet(new HashSet<I>(alphabet))) {
		if(!s.isEmpty()) {
			for(List<I> permutation : com.google.common.collect.Collections2.permutations(s)) {
				if(!hypothesis.query(permutation).equals(this.membershipQuery(permutation))) {
					O a = hypothesis.query(permutation);
					O b = this.membershipQuery(permutation);
					return permutation;
				}
			}
		}
	}
	return null;
}
\end{lstlisting}


